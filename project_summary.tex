\documentclass[12pt,a4paper]{article}
\usepackage[UTF8]{ctex}
\usepackage{geometry}
\usepackage{graphicx}
\usepackage{booktabs}
\usepackage{longtable}
\usepackage{array}
\usepackage{xcolor}
\usepackage{listings}
\usepackage{hyperref}
\usepackage{fontawesome5}
\usepackage{tikz}
\usepackage{tcolorbox}
\usepackage{enumitem}
\usepackage{fancyhdr}
\usepackage{titlesec}

% 页面设置
\geometry{left=2.5cm, right=2.5cm, top=2.5cm, bottom=2.5cm}

% 颜色定义
\definecolor{primaryblue}{RGB}{41, 128, 185}
\definecolor{secondarygreen}{RGB}{39, 174, 96}
\definecolor{warningorange}{RGB}{230, 126, 34}
\definecolor{codebg}{RGB}{245, 245, 245}
\definecolor{codeframe}{RGB}{200, 200, 200}

% 代码样式
\lstset{
    basicstyle=\ttfamily\small,
    backgroundcolor=\color{codebg},
    frame=single,
    framerule=0.5pt,
    rulecolor=\color{codeframe},
    breaklines=true,
    breakatwhitespace=true,
    showstringspaces=false,
    tabsize=4,
    captionpos=b,
    numbers=none,
    xleftmargin=10pt,
    xrightmargin=10pt,
    aboveskip=10pt,
    belowskip=10pt
}

% 标题样式
\titleformat{\section}{\Large\bfseries\color{primaryblue}}{\thesection}{1em}{}
\titleformat{\subsection}{\large\bfseries\color{primaryblue!80}}{\thesubsection}{1em}{}

% 页眉页脚
\pagestyle{fancy}
\fancyhf{}
\fancyhead[L]{\textcolor{gray}{新工科 AI 助教系统}}
\fancyhead[R]{\textcolor{gray}{项目技术总结}}
\fancyfoot[C]{\thepage}
\renewcommand{\headrulewidth}{0.4pt}

% tcolorbox 样式
\tcbuselibrary{skins,breakable}

\newtcolorbox{highlightbox}[1][]{
    colback=primaryblue!5,
    colframe=primaryblue,
    fonttitle=\bfseries,
    title=#1,
    breakable
}

\newtcolorbox{tipbox}[1][]{
    colback=secondarygreen!5,
    colframe=secondarygreen,
    fonttitle=\bfseries,
    title=#1,
    breakable
}

\newtcolorbox{warningbox}[1][]{
    colback=warningorange!5,
    colframe=warningorange,
    fonttitle=\bfseries,
    title=#1,
    breakable
}

\begin{document}

% 封面
\begin{titlepage}
    \centering
    \vspace*{2cm}
    
    {\Huge\bfseries\color{primaryblue} 🎓 新工科 AI 助教系统}
    
    \vspace{0.5cm}
    {\Large\color{gray} 项目技术总结报告}
    
    \vspace{2cm}
    
    \begin{tcolorbox}[
        colback=white,
        colframe=primaryblue,
        width=0.8\textwidth,
        arc=5mm
    ]
        \centering
        \large
        基于 \textbf{RAG-Anything} 框架构建的\\[0.5em]
        多模态智能问答系统
    \end{tcolorbox}
    
    \vspace{2cm}
    
    \begin{tabular}{rl}
        \textbf{文本模型:} & DeepSeek (deepseek-chat) \\[0.5em]
        \textbf{视觉模型:} & 阿里通义 Qwen-VL-Max \\[0.5em]
        \textbf{嵌入模型:} & BGE-Small-ZH (本地) \\[0.5em]
        \textbf{RAG框架:} & LightRAG + 知识图谱 \\
    \end{tabular}
    
    \vfill
    
    {\large \today}
    
\end{titlepage}

% 目录
\tableofcontents
\newpage

% ========== 正文 ==========

\section{项目概述}

本项目基于 \textbf{RAG-Anything} 框架构建了一个多模态 AI 助教系统,能够:

\begin{itemize}[leftmargin=2em]
    \item 解析 PDF 教材中的\textbf{文本、图片、表格和公式}
    \item 构建\textbf{知识图谱}进行语义检索
    \item 提供\textbf{个性化智能问答}功能
    \item 支持\textbf{多图片批量处理}
\end{itemize}

\begin{highlightbox}[核心特性]
\begin{itemize}
    \item \textbf{双模型协作}:文本处理使用 DeepSeek,图像理解使用 Qwen-VL
    \item \textbf{智能分批}:自动将大量图片分批处理,突破 API 限制
    \item \textbf{优雅降级}:VLM 调用失败时自动回退到纯文本模式
\end{itemize}
\end{highlightbox}

\section{大模型架构}

\subsection{双模型协作设计}

系统采用\textbf{文本模型 + 视觉模型}的双模型协作架构:

\begin{center}
\begin{tikzpicture}[
    node distance=1.5cm,
    box/.style={rectangle, draw=primaryblue, fill=primaryblue!10, 
                text width=3cm, text centered, rounded corners, minimum height=1cm},
    arrow/.style={->, thick, primaryblue}
]
    % 节点
    \node[box] (query) {用户查询};
    \node[box, below of=query] (rag) {RAG 检索\\(LightRAG)};
    \node[box, below left=1.5cm and 0.5cm of rag] (text) {文本处理\\DeepSeek};
    \node[box, below right=1.5cm and 0.5cm of rag] (vision) {图像处理\\Qwen-VL};
    \node[box, below=3.5cm of rag] (output) {综合回答生成};
    
    % 箭头
    \draw[arrow] (query) -- (rag);
    \draw[arrow] (rag) -- (text);
    \draw[arrow] (rag) -- (vision);
    \draw[arrow] (text) -- (output);
    \draw[arrow] (vision) -- (output);
\end{tikzpicture}
\end{center}

\subsection{模型选型}

\begin{table}[htbp]
\centering
\caption{模型配置详情}
\begin{tabular}{@{}llll@{}}
\toprule
\textbf{功能} & \textbf{模型} & \textbf{提供商} & \textbf{特点} \\
\midrule
文本理解/生成 & deepseek-chat & DeepSeek & 中文能力强、性价比高 \\
图像理解 & qwen-vl-max & 阿里云 & 支持多图(10张)、中文OCR强 \\
文本嵌入 & bge-small-zh-v1.5 & 本地 & 512维向量、离线可用 \\
\bottomrule
\end{tabular}
\end{table}

\section{技术亮点}

\subsection{亮点一:智能图片分批处理}

当 PDF 中的图片数量超过 VLM 单次请求限制时,系统自动进行分批处理:

\begin{lstlisting}[language=Python, caption=分批处理流程]
# 当图片超过单次限制时自动分批
# 检测到 25 张图片,每批最多 10 张
# 将 25 张图片分成 3 批处理
# 正在处理第 1/3 批...
# 正在处理第 2/3 批...
# 正在处理第 3/3 批...
# 正在用 DeepSeek 综合所有批次的分析结果...
\end{lstlisting}

\begin{tipbox}[解决的问题]
突破 VLM API 的单次图片数量限制,支持任意数量图片的完整处理。每个批次的分析结果最终由 DeepSeek 综合成完整答案。
\end{tipbox}

\subsection{亮点二:DeepSeek 兼容性适配}

LightRAG 框架默认使用 OpenAI 的 \texttt{response\_format} 结构化输出功能,但 DeepSeek 不支持此功能。

\begin{lstlisting}[language=Python, caption=兼容性处理]
# 自动移除 DeepSeek 不支持的参数
kwargs.pop('response_format', None)
kwargs.pop('keyword_extraction', None)
\end{lstlisting}

\textbf{修改位置:}
\begin{itemize}
    \item \texttt{app.py} 中的 \texttt{safe\_deepseek\_call} 和 \texttt{vision\_func}
    \item \texttt{lightrag/llm/openai.py} 中的 DeepSeek 检测逻辑
\end{itemize}

\subsection{亮点三:优雅的错误回退机制}

\begin{lstlisting}[language=Python, caption=错误回退逻辑]
try:
    # 尝试调用 VLM
    response = await openai_complete_if_cache(vision_model, ...)
except Exception as e:
    # 失败时自动回退到纯文本模式
    print(f"VLM 调用失败: {e},回退到纯文本模式")
    return await safe_deepseek_call(text_prompt, ...)
\end{lstlisting}

\begin{warningbox}[解决的问题]
VLM 调用失败时不会中断用户体验,自动降级到纯文本问答,确保系统稳定性。
\end{warningbox}

\subsection{亮点四:动态模型配置}

根据配置的视觉模型提供商自动调整图片数量限制:

\begin{lstlisting}[language=Python, caption=动态配置逻辑]
if vision_provider == "qwen":
    MAX_IMAGES_PER_REQUEST = 10   # 阿里 Qwen-VL
elif vision_provider == "siliconflow":
    MAX_IMAGES_PER_REQUEST = 10   # 硅基流动
else:
    MAX_IMAGES_PER_REQUEST = 1    # 智谱 GLM-4V
\end{lstlisting}

\subsection{亮点五:个性化教学模式}

系统根据用户选择的学习水平调整回答风格:

\begin{table}[htbp]
\centering
\caption{教学模式对比}
\begin{tabular}{@{}lll@{}}
\toprule
\textbf{模式} & \textbf{目标用户} & \textbf{回答风格} \\
\midrule
直觉科普模式 & 初学者 & 用生活类比解释概念 \\
标准教学模式 & 本科生 & 定义→公式→物理意义→考点 \\
深度研讨模式 & 专家 & 学术简练、深入机制 \\
\bottomrule
\end{tabular}
\end{table}

\section{项目配置}

\subsection{环境变量配置 (.env)}

\begin{lstlisting}[caption=.env 配置文件]
# 文本模型 (DeepSeek)
LLM_BINDING_HOST=https://api.deepseek.com/v1
LLM_BINDING_API_KEY=sk-xxx
OPENAI_API_KEY=sk-xxx

# 视觉模型 (阿里 Qwen-VL)
VISION_PROVIDER=qwen
QWEN_API_KEY=sk-xxx
\end{lstlisting}

\subsection{支持的视觉模型}

\begin{table}[htbp]
\centering
\caption{可选视觉模型}
\begin{tabular}{@{}lllll@{}}
\toprule
\textbf{提供商} & \textbf{模型} & \textbf{多图支持} & \textbf{配置} \\
\midrule
智谱 & glm-4v & 1张 & \texttt{VISION\_PROVIDER=zhipu} \\
\textbf{阿里} & qwen-vl-max & \textbf{10张} & \texttt{VISION\_PROVIDER=qwen} \\
硅基流动 & Qwen2-VL-72B & 10张 & \texttt{VISION\_PROVIDER=siliconflow} \\
\bottomrule
\end{tabular}
\end{table}

\section{关键修改文件}

\begin{table}[htbp]
\centering
\caption{修改文件清单}
\begin{tabular}{@{}lp{8cm}@{}}
\toprule
\textbf{文件} & \textbf{修改内容} \\
\midrule
\texttt{app.py} & 双模型架构、图片分批处理、错误回退机制、动态配置 \\
\texttt{lightrag/llm/openai.py} & DeepSeek \texttt{response\_format} 兼容性修复 \\
\texttt{.env} & API 密钥和模型配置 \\
\bottomrule
\end{tabular}
\end{table}

\section{运行指南}

\subsection{启动命令}

\begin{lstlisting}[language=bash]
cd C:\RAG-Anything\RAG-Anything
.\venv\Scripts\activate
streamlit run app.py
\end{lstlisting}

\subsection{访问地址}

\begin{itemize}
    \item 本地访问: \url{http://localhost:8501}
    \item 局域网访问: \url{http://你的IP:8501}
\end{itemize}

\section{系统架构图}

\begin{center}
\begin{tikzpicture}[
    node distance=0.8cm,
    layer/.style={rectangle, draw=primaryblue, fill=primaryblue!10, 
                  text width=12cm, text centered, rounded corners, minimum height=1.2cm},
    component/.style={rectangle, draw=gray, fill=gray!10, 
                      text width=3.5cm, text centered, rounded corners, minimum height=0.8cm}
]
    % 层级
    \node[layer] (ui) {\textbf{Streamlit Web UI}\\PDF上传 | 问答输入 | 学习模式选择};
    
    \node[layer, below=0.5cm of ui] (rag) {\textbf{RAGAnything 框架}\\MinerU 解析 | LightRAG 知识图谱 | 多模态处理器};
    
    \node[component, below left=0.8cm and 1cm of rag] (deepseek) {\textbf{DeepSeek}\\文本处理\\实体提取\\答案生成};
    
    \node[component, below right=0.8cm and 1cm of rag] (qwen) {\textbf{Qwen-VL-Max}\\图像处理\\分批处理\\内容理解};
    
    % 连接
    \draw[->, thick, primaryblue] (ui) -- (rag);
    \draw[->, thick, primaryblue] (rag) -- (deepseek);
    \draw[->, thick, primaryblue] (rag) -- (qwen);
\end{tikzpicture}
\end{center}

\section{总结}

本项目成功实现了一个功能完整的多模态 AI 助教系统,主要技术贡献包括:

\begin{enumerate}[leftmargin=2em]
    \item \textbf{双模型协作架构}:DeepSeek 负责文本,Qwen-VL 负责图像
    \item \textbf{智能分批处理}:突破 VLM 图片数量限制
    \item \textbf{完善的错误处理}:优雅降级确保系统稳定
    \item \textbf{灵活的配置机制}:支持多种 VLM 提供商切换
    \item \textbf{个性化教学}:根据用户水平调整回答风格
\end{enumerate}

\vspace{1cm}
\begin{center}
\textcolor{primaryblue}{\Large 🎉 项目成功运行!}
\end{center}

\end{document}
